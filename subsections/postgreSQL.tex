Die Datenbank wurde um 1986 an der Universität von Kalifornien in Berkeley entwickelt.
Die Software ist eine Open-Source Objekt-RDBMS. Das System gewährleistet die Datenintegrität 
unabhängig von der Datengröße und ist zudem in hohem Maße erweiterbar, z. B. mit benutzerdefinierten
Datensätzen, Funktionen, sekundäre Datenbankmodellen und sogar verschiedenen Programmiersprachen, 
ohne dass die Datenbank neu kompiliert werden muss. PostgreSQL ist eine komplexere Datenbank als
SQLite in Bezug auf Funktionen, unterstützte Datentypen und Zuverlässigkeit. \cite{PostgreSQL_About}

\begin{large} \emph{\textbf{Hauptmerkmalen}} \end{large}
\begin{itemize}
    \item \underline{\texttt{Portabilität}}:
        PostgreSQL ist nur portabel, wenn die Datenbank in eine Datei exportiert und dann
        auf einen Server kopiert wird. \cite{SQLite_vs_PostgreSQL}
    \item \underline{\texttt{Unterstützte Datensätze}}:
        Das Programm kann jeden Datentyp speichern, der in einer Datenbank vorkommen
        kann. \cite{SQLite_vs_PostgreSQL}
    \item \underline{\texttt{Datenbankmodell}}:
        die PostgreSQL-Datenbank benötigt mehr Speicherplatz (200 MB), da sie ein 
        Client-Server-Modell darstellt und einen Datenbankserver zur Installation und 
        Ausführung benötigt.\cite{SQLite_vs_PostgreSQL}
    \item \underline{\texttt{Sicherheit}}:
        PostgreSQL bietet mehrere Datensicherheitsmechanismen, z.B. GSSAPI, SSPI, LDAP, 
        um sensible Daten zu schützen.\cite{PostgreSQL_Authentication}
    \item \underline{\texttt{Standardisiert}}:
         PostgreSQL unterstützt 160 der 190 grundlegenden SQL-Features.\cite{PostgreSQL_standardisierung}
    \item \underline{\texttt{Backup}}:
        PostgreSQL bietet drei verschiedene Backup-Strategien: SQL Dump, File System 
        Level Backup und kontinuierliche Archivierung.\cite{PostgreSQL_Backup}
\end{itemize} 
\begin{large} \emph{\textbf{Fazit}} \end{large}
PostgreSQL verfügt über zusätzliche Funktionen, die für Integrität und Sicherheit sorgen,
sowie über die Portabilität, um komplexe Datenbanken problemlos zu betreiben. Der Preis
dafür ist ein erhöhter anfänglicher Ressourcenbedarf, sowohl in Bezug auf den 
Arbeitsspeicher als auch auf den Speicherplatz, und eine umständlichere Implementierung. 