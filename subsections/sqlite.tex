Diese Datenbank wurde von Dwayne Richard Hipp in den frühen 2000er Jahren entwickelt und bietet ein relationales,
leichtgewichtige Datenbankmanagementsystem ohne sekundäre Datenbankmodelle. Einer der Hauptvorteile ist die Tatsache,
dass es sich um eine Open-Source-Datenbank für mehrere Plattformen (Windows, Mac, Linux, Unix usw.) handelt,
die dank der einfachen Datenspeicherung,
der dynamischen Datentypen und einer breiten Palette von Programmiersprachen (VS Basic, C-sharp, PHP, Python usw.)
eine schnelle und einfache Lösung für die Speicherung von Daten bietet.\cite{SQLite_wiki}


\begin{large} \emph{\textbf{Hauptmerkmalen}} \end{large}
\begin{itemize}
    \item \underline{\texttt{Portabilität}}:
        SQLite speichert seine Datenbank in einer normalen Festplattendatei 
        innerhalb des Verzeichnisses,
        was es zu einem der portabelsten RDBMS macht. \cite{Was_ist_SQLite}
    \item \underline{\texttt{Unterstützte Datensätze}}:
        die grundlegenden Datentypen sind 64-Bit-Integer, 64-Bit-IEEE-Gleitkommazahl,
        String, TEXT, BLOB, NULL. \cite{SQLite_datatypes}
    \item \underline{\texttt{Eigenständig}}:
        nicht vom Betriebssystem oder externen Bibliotheken abhängig und kann aufgrund seines geringen
        Speicherplatzbedarfs (500KB) in einer Vielzahl von Umgebungen eingesetzt werden, einschließlich 
        mobiler Geräte (Android, iPhone) und Spielkonsolen.\cite{SQLite_features}
    \item \underline{\texttt{Geschwindigkeit}}:
        SQLite schneidet bei den folgenden Abfragen gut ab (insbesondere, wenn sie in einer Transaktion sind),
        dank seiner kompakten, einfachen Architektur und Gestaltung. Siehe Anhang \ref{geschwindigkeit_der_querys} . \cite{SQLite_speed}
    \item \underline{\texttt{Serverlos}}:
        keine Client-Server-Architektur ist erforderlich, so dass es innerhalb der Anwendung arbeiten kann. \cite{SQLite_serverless}
    \item \underline{\texttt{Zero Konfiguration}}:
        keine Installations- oder Serveranforderungen von dem Benutzer erforderlich und es ohne Datenbankkonfiguration
        verwendet werden kann.\cite{SQLite_features}
    \item \underline{\texttt{Transaktional}}:
        Transaktionen folgen der A-C-I-D-Struktur für Zuverlässigkeit, was bedeutet, dass die meisten Abfrageeigenschaften
        den Grundsätzen der Atomarität, Konsistenz, Isolierung und Dauerhaftigkeit entsprechen.\cite{SQLite_features}  
    \item \underline{\texttt{Online Backup-API}}:
        Die Online-Backup-API, der einzige Datenwiederherstellungsmechanismus, der es ermöglicht, Daten von einer Datenbank
        in eine andere zu kopieren, sogar inkrementell, indem der Inhalt der Zieldatenbank unter Verwendung von Shared-Locks überschrieben wird.\cite{SQLite_backup}
\end{itemize} 
\begin{large} \emph{\textbf{Fazit}} \end{large}
    SQLite ist eine leicht einzurichtende, schnelle und benutzerfreundliche Datenbank, deren Schwäche jedoch ihre Einfachheit ist.
    Mit zunehmender Komplexität der Aufgaben werden Unzulänglichkeiten wie die Verarmung der Datentypen, fehlende Sicherheitsmechanismen
    für Daten und Backups, nicht unterstützte Dateiformate (XML), fehlender Mehrfachzugriff und fehlende erweiterte Funktionen immer deutlicher.  