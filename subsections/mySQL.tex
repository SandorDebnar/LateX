MYSQL wurde erstmals 1944 von einem schwedischen Unternehmen entwickelt und die erste Version
wurde 1945 veröffentlicht. Das Unternehmen wurde in 2008 von Sun Microsystem und in 2010 von seinem
heutigen Eigentümer Oracle übernommen.\cite{MySQL_Wiki}

MySQL ist ein reines RDBMS, das in seinen kostenpflichtigen Paketen neben den Basisdiensten 
zusätzliche Funktionen und die neuesten technologischen Entwicklungen bietet. 

\begin{large} \emph{\textbf{Hauptmerkmalen}} \end{large}
\begin{itemize}
    \item \underline{\texttt{Portabilität}}:
        Die in C und C++ geschriebene Software funktioniert nicht nur in Umgebungen,
        die mit verschiedenen Compilern getestet wurden, sondern auch auf einer Vielzahl 
        von Betriebssystemen und kann problemlos mit der Software CMake portiert werden.\cite{MySQL_Features}
    \item \underline{\texttt{MVCC}}:
        Die Datenbankfunktionen folgen der MVCC-Architektur (Multi-Version Concurrency Control),
        so dass Datenbankzugriffe mit gleicher Priorität effizient durchgeführt werden können, 
        ohne die Datenbank zu blockieren oder ihre Konsistenz zu gefährden.\cite{MVCC_Wiki}
    \item \underline{\texttt{Unterstützte Datensätze}}:
        Die Software unterstützt nicht nur die meisten Datentypen, sondern bietet auch spezielle 
        Funktionen wie das Referenzieren von Tabellen in anderen Datenbanken.\cite{MySQL_Features}
    \item \underline{\texttt{Sicherheit}}:
        MySQL bietet auch Sicherheitsmechanismen zum Schutz der Daten. 
        (SHA-2, PAM, LDAP, Kerberos usw.) \cite{MySQL_Authentication}
    \item \underline{\texttt{Storage Engines}}:
        Sind Komponenten, die vom System für verschiedene Operationen verwendet werden kann.
        Die unterstützten Engines sind InnoDB, MyISAM, Memory, CSV, Archive, Blackhole, NDB,
        Merge, Federated, Example. \cite{MySQL_Storage_Engines}
    \item \underline{\texttt{Servereinstellungen}}:
        Die Datenbank bietet die Möglichkeit, Festplatten I/O, Speichernutzung, Dateilayout und 
        Systemfaktoren wie External Locks zu optimieren.\cite{MySQL_Server_Optimalization}
    \newpage
    \item \underline{\texttt{Backup und Recovery}}:
        MySQL bietet eine Reihe von Varianten für Backup-Strategien, wie z. B. physische und logische
        Backups, Online- und Offline-Backups, lokale und Remote-Backups, Snapshot-Backups sowie
        vollständige und inkrementelle Backups.\cite{MySQL_Backup}
\end{itemize} 
\begin{large} \emph{\textbf{Fazit}} \end{large}
MySQL ist ein einfaches RDBMS mit den meisten Funktionen, die für moderne Datenbanken erforderlich 
sind, sowohl in Bezug auf Sicherheit, Integrität und Flexibilität. Häufige Aktualisierungen und neue
Funktionen sind nicht nur für kostenpflichtige Pakete verfügbar, so dass es sich wirklich um eine
kostenlose Open-Source-Datenbank handelt. 