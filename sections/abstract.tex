\newpage
\begin{spacing}{1}
    \chapter*{Zusammenfassung}
\end{spacing}


Die Automatisation und Zentralisation der Eingabe von Informationen trägt dazu bei, die Organisation verschiedener Aufgaben und Tätigkeiten schnell,
effizient und ressourcenschonend zu erledigen. 

Die HTL Leonding ist jedes Jahr mit der Organisation der Modulanmeldungen und der Schulbesuchs- und AMS-Bestätigungen konfrontiert. 

Das Ziel der vorliegenden Arbeit ist es, eine Web-Anwendung zu implementieren, die die organisatorischen Aufgaben der HTL Leonding digitalisiert.
Dabei spielen die Wahl des Frameworks und des Datenbanksystems eine wichtige Rolle. Verschiedene Systeme wurden auf Performance, Funktionalität und Beliebtheit bei Entwicklern untersucht. 

Die vorliegende Software wurde mit dem Angular Front-End Webapplikations-Framework und einer Oracle SQL-Datenbank im Back-End entwickelt. 

Durch die Software wurde nicht nur die Datenverarbeitung beschleunigt, sondern auch der Datenbestand übersichtlicher gestaltet und für zukünftige statistische Auswertungen aufbereitet.


\begin{spacing}{1}
    \chapter*{Abstract}
\end{spacing}
The automation and centralization of the input of information helps to organize various tasks and activities quickly,
efficiently and in a resource-saving manner. 
HTL Leonding is confronted with the organization of module registrations and school attendance and AMS confirmations every year. 

The goal of this thesis is to implement a web application that digitizes the organizational tasks of HTL Leonding.
The choice of the framework and the database system play an important role. Different systems were tested for performance,
functionality and popularity with developers. 
The present software was developed with the Angular front-end web application framework and an Oracle SQL database in the back-end. 
The software not only accelerated data processing, but also made the dataset clearer and prepared it for future statistical analysis.



