\section{Abendschule}
\setauthor{Emir Bajramovic}

Seit dem Schuljahr 2015/2016 gibt es die Möglichkeit in Leonding eine Abendschule bzw. ein Kolleg in Informatik zu besuchen.
Der Lehrplan sieht eine Ausbildung in der Informatik vor. 
Als Schwerpunkt wird an der HTBLA-Leonding Software-Engineering angeboten. 
Abhängig davon ob man die Berufsreifeprüfung schon abgelegt hat, meldet man sich entweder für die Abendschule oder für das Kolleg an. 
In der Abendschule dauert die Ausbildung 8 Semester mit vorgezogener Reifeprüfung nach 2 Jahren und endet mit der Reife- und Diplomprüfung.
Besucht man jedoch das Kolleg für Berufstätige für Informatiker, dann dauert die Ausbildung lediglich 6 Semester. 
Die allgemeinbildenden Gegenstände fallen weg und man schließt die Ausbildung mit der Diplomprüfung ab.




\section{Ist-Situation}
\setauthor{Emir Bajramovic}

In der HTBLA-Leonding werden derzeit die Modulanmeldungen noch per Stift und Papier getätigt. 
Die Schüler können sich mit einem X neben ihrem Namen und unter dem jeweiligen Fach für das kommende Semester eintragen. 
Hierbei steht ein „B“ für befreit, falls dem Abendschüler ein Fach angerecht wurde und dieser freigestellt ist. 
Die Liste mit den Modulanmeldungen wird dann im Sekretariat abgelegt und weiterbearbeitet. 
Nun hat uns Herr Professor Stöttinger gebeten über einen Lösungsansatz nachzudenken. 

